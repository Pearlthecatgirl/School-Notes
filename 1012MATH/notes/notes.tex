\documentclass[11pt]{book}
\usepackage[margin=2cm]{geometry}
\usepackage{amsmath}
\usepackage{amssymb}
\author{pearl}
\title{MATH1012 Notes}

\begin{document}
\maketitle
\chapter{Linear Algebra}
\section{Set of linear equations}
\subsection{Definitions}
\begin{itemize}
	\item{Linear equation: is an equation whose terms are numbers, and variables to the first power e.g $x+y=3$ is linear, but  $x^{2}+4y=5$ isn't.}
	\item{Solution: a solution is a choice of values (numbers) that satisfies a linear equation.}
	\item{System of linear Equation: is just a collection of linear equation. A solution to a system linear equation is a set of solutions which satisfy all equations. Give solution using set notation. E.g: $S=\{\left(2,1\right)\}$ for the graph: $x+2y=4, x-y=1$ an SLE in 2 dimensional space can either have:
	\begin{itemize}
		\item{No solutions (inconsistent): e.g 2 parallel lines}
		\item{unique solution: 2 intersecting lines}
		\item{Infinitely many solutions: 2 lines are the same line/overlay-ed on top of each other}
	\end{itemize}
	}
\end{itemize}
\subsubsection{Solving set of linear equations}
\par{Solve them geometrically or solve them through algebra. Example below:}
\begin{align*}
	x+y+z=1 \\ 
	x+y-z=1
\end{align*}
\subsubsection{Solving geometrically}
\begin{enumerate}
	\item{guess 1 point from each equation: (1,0,0) and (0,1,0)}
	\item{The line that connects up the points is a solution}
\end{enumerate}
\subsubsection{Solving algebraically}
\begin{enumerate}
	\item{Subtraction method. 'Subtract' one equation from the other
		\begin{align*}
			(x+y+z=1) - (x+y-z=1)\\
			\implies 0 + 0 + 2z=0 \\
			\implies z=0 \\
			\implies x+y+0=1 \\
			\implies x+y=1
		\end{align*}
		}
\end{enumerate}
\subsection{Parametric Solutions}
\par{In a parametric solution, every coordinate is either \textbf{constant, free variable, linear combination }(constants and free variables). i.e expressing the equation in terms of one variable. We can take x as the free variable, or y. Using the above as an example:}
\begin{align*}
	x+y+z=1 \\ 
	x+y-z=1 \\
	\implies x+y=1 \\
	\implies x=1-y \\
	\implies y=1-x \\
	\implies z=0 \\
	\therefore S=\{\left(x, 1-x, 0\right)\ \|x\in \mathbb{R} \}
\end{align*}
\par{Some variables cannot be chosen as free variables, in this example: z.}
\subsection{Elementary Row Operations}
\par{Given a set of linear equations: 
\begin{align*}
	x+y=5 \\
	2x-2y=7
\end{align*}
We can do the following and have no effect on the solution set:
\begin{itemize}
	\item{swap 2 equations' locations}
		\begin{align*}
			2x-2y=7 \\
			x+y=5
		\end{align*}
	\item{Multiply an equation by a non-zero number}
		\begin{align*}
			x+y=5 \\
			\text{becomes} \\
			2x+2y=2 \times 5 \\
			= 2x +2y=10
		\end{align*}
	\item{Add a multiple of one equation to another}
		\begin{align*}
			x+y=6 \\
			2x+2y=10 \\
		\end{align*}
		$R2\leftarrow R2-R1$
		\begin{align*}
			x+y=6 \\
			-4y=-3 \\
			y=\frac{-3}{-4} \\
			y=\frac{3}{4}
		\end{align*}
\end{itemize}
}
\subsection{Matrix notation and Gaussian elimination}
\par{Systems of linear equations can be expressed as a matrix, where the variables (x,y,z , etc) are omitted and only the coefficients are written down. Example:
\begin{align}
	x+2y=7 \\
	2x-y=4
\end{align}
Becomes:
\[
\left[
\begin{array}{cc|c}
	\text{x column} & \text{y column} & \text{solution} \\
	1 & 2 & 7 \\
	2 & -1 & 4
\end{array}
\right]
\]
\subsection{Row Echelon Form}
\par{A matrix is in row echelon form if:
	\begin{itemize}
		\item{The rows that are \textbf{all-zero} should be at the bottom}
		\item{The \textbf{leading entry} of every other row is \textbf{further to the right} than the leading entry of any previous rows}
		\item{The leading entry is the first number of each row}
		\item{\textbf{basic variables} are variables that are also leading entries.}
		\item{All non-basic variables are free variables}
	\end{itemize}
}
\subsubsection{Back substitution}
\par{Read the last row and it should give one of the answers. E.g We want to have a set of answers which are either:
\begin{itemize}
	\item{free variables (non basic)}
	\item{numbers}
	\item{a combination of both}
\end{itemize}}
\subsubsection{Gaussian Elimination}
\par{Example: }
\begin{itemize}
	\item{The pivot is the top row. We want to apply row operations to make the numbers below the pivot to 0}
\end{itemize}
\[
\left[
\begin{array}{cccc|c}
1 & 0 & 1 & 3 & 2 \\
2 & 1 & 2 & 0 & 0 \\
-1 & 3 & 4 & 0 & 1
\end{array}
\right]
\]
\begin{enumerate}
	\item{R2$\leftarrow$ R2-2R1}
	\item{R3$\leftarrow$ R3-2R1}
\end{enumerate}
\[
\left[
\begin{array}{cccc|c}
1 & 0 & 1 & 3 & 2 \\
0 & 1 & 0 & -6 & -4 \\
0 & 3 & 5 & 3 & 3
\end{array}
\right]
\]
\begin{enumerate}
	\item{R3 $\leftarrow$ R3 - 3 $\times$ R2}
\end{enumerate}
\[
\left[
\begin{array}{cccc|c}
1 & 0 & 1 & 3 & 2 \\
0 & 1 & 0 & -6 & -4 \\
0 & 0 & 5 & 21 & 15
\end{array}
\right]
\]
\par{Here we are now in reduced row echelon form. We can see that the $x_{4}$ column is non-basic. We want to express the basic variables in terms of the non basic variables.}
\subsubsection{Counting solutions}
\par{To determine whether the solution count is 0, 1 or $\infty$}
\begin{itemize}
	\item{If all numbers are 0 producing a non-zero, then there are no solutions}
	\item{if there is a free variable there are infinite solutions}
	\item{if there are no free variables and there doesn't exist a row containing all zeros producing a non-zero value in the augmented column then there is exactly 1 solution.}
\end{itemize}
\begin{itemize}
	\item{if SLE is \textbf{inconsistent then} $S =\emptyset$}
	\item{If SLE has a \textbf{unique solution}, then get S from the matrix}
	\item{if there are infinitely many solutions, then use back substitution to express S in \textbf{parametric form}}
	\item{S is the set of solutions}
\end{itemize}
\subsubsection{Reduced Row Echelon Form}
\par{There is a simpler form of matrices called reduced row echelon form. It has the extra condition from regular row echelon form in that both above and below the pivot row has to be 0. The leading entries must also be '1'. This form lets us get the direct parametric form in in the solution for back substitution.}
\begin{enumerate}
	\item{Ensure the entry in pivot position is non-zero (swap rows if necessary)}
	\item{Add a suitable multiple of pivot row to each row above and below the pivot row to zero out the matrix entries above and below the pivot}
	\item{Move the pivot position 1 column and 1 row}
\end{enumerate}
\section{Vector spaces and subspaces}
\begin{itemize}
	\item{The set of all vectors of \textbf{arity} n is denoted $\mathbb{R}^{n}$}
	\item{adding vectors of the same arity}
	\item{multiplying vectors by a number}
	\item{Set of solutions of SLE is just a vector}
\end{itemize}
\subsection{Closure}
\par{A set of vectors that are closed under addition if any vector from that set added to another vector in that set will produce a vector that exists within its original set. This also applies to multiplication: a vector multiplied by a number will stay in its original set. Closed in general is a property of a set of vectors for which when an operation is applied to such set, the solution is part of the set. This property must hold for all members of the set. E.g:
\begin{align*}
	S=\{\left(1,2\right), \left(2,4\right), \left(3,6\right)\} \\
	\text{The above is not closed as not all pairs produce addition answers within the set}
\end{align*}
in order to show closure, provide a specific counterexample that always works, or symbolic proof. THe proof must work for all examples, whereas the disproof only needs to show 1 counterexample.}
\subsection{Subspace}
\par{A set of vectors is a subspace if
\begin{itemize}
	\item{The zero vector must be contained in the set.}
	\item{The set must be closed under addition/multiplication.}
\end{itemize}
Testing if a subspace does not contain a zero, is a fast way to test if a subspace is valid. In $\mathbb{R}^{2}$, valid subspaces are typically the full set, the origin, or a line through the origin. E.g:
\begin{itemize}
	\item{$x=y$ in $\mathbb{R}^{2}$}
	\item{$x+1=2y$ in $\mathbb{R}^{2}$}
\end{itemize}
\subsubsection{Proving $x=y$}
\begin{align*}
	x+2y-3z=0 \\
	(0)+2(0)-3(0)=0 \\
	\implies (0,0,0) \text{is within the set} \\
	\text{let } u,v \in S\\
	\text{then} \left(u_{1}+u_{1},u_{2}+u_{2},u_{3}+u_{3}\right) \\
	u_{1}+v_{1}+2\left(u_{1}+v_{1}\right)-3\left(u_{1}+v_{1}\right)
\end{align*}
}

\subsubsection{Proving the validity of subspaces}
\par{The best way to prove the validity of a subspace is to check the 3 criteria as listed above.}
\begin{itemize}
	\item{The origin, or zero vector $\left(0,0,0....0_{n}\right)$ is included for $S\in\mathrm{R}^{n}$. This is often the easiest check}
	\item{$S$ must be closed under addition. This means if you take 2 vectors within the subset, the result must still be within the subspace.}
	\item{$S$ must be closed under scalar multiplication. This means if you take a vector in the set and multiply it by any scalar, it must stay within the set.}
\end{itemize}
\subsubsection{How to check each condition}
\par{Example: Given $S\in\mathrm{R^{3}}, x+2y-3z=0$}
\begin{enumerate}
	\item{The first step is to decide whether the origin/$\vec{0}$ is in the subset. The easiest way to check is to substitute the $\vec{0}$ of the current arity into the linear equation and see if it holds up. \textbf{Here the 0 vector is in the set because:}}
\begin{align*}
	& \left(0,0,0\right)\rightarrow x+2y-3z=0 \\
	& \left(0\right)+2\left(0\right)-3\left(0\right)=0 
\end{align*}
\item{Next we must show that if we take any 2 solutions, and we add them together, we get another solution within the set. Start by defining 2 elements within the set, let $\vec{u}=\left(u_{1},u_{2},u_{3}\right)\in S$ and let $\vec{v}=\left(v_{1},v_{2},v_{3}\right)\in S$ We do not know anything about u or v, except they are both vectors that exist in $S$. They must exist in  $S$ as that is they are hypothetical solutions of S.}
	\begin{enumerate}
		\item{ let: $\vec{u}=\left(u_{1},u_{2},u_{3}\right)\in S$ and let $\vec{v}=\left(v_{1},v_{2},v_{3}\right)\in S$ }
		\item{ then: $u_{1}+u_{2}-3u_{3}=0$ and $v_{1}+v_{2}-3v_{3}=0$  due to $\vec{u}$ and $\vec{v}\in S$ }
		\item{ The above line shows that the 2 vectors we are using are in the set of solutions of $S$ }
		\item{ Check if: $\vec{u}+\vec{v}=\left(u_{1}+v_{1},u_{2}+v_{2},u_{3}+v_{3}\right)$ Gives the solution of the equation we are testing}
		\item{ Demonstrate the following is equal to zero, but do not state it beforehand.}
		\item{ $\vec{u}+\vec{v}=u_{1}+v_{1}+2\left(u_{2}+v_{2}\right)-3\left(u_{3}+v_{3}\right)$}
		\item{ After rearranging, we get: $\vec{u}+\vec{v}=u_{1}+u_{2}-3u_{3}+v_{1}+v_{2}-3v_{3}$}
		\item{ Which we determined in step 2 to be $\vec{u}+\vec{v}=0+0=0$}
		\item{ We have deduced that both u and v are in the set, and that u+v are also in the set, therefore it is closed under addition.}
	\end{enumerate}
	\item{Next we must show that if we take any scalar value and multiply it by any vector, in this case we can use $u$ from the previous step, as we have already proved it is in the set  $S$, that it remains within the set of solutions. }
	\begin{enumerate}
		\item{let $\alpha$ be any scalar multiple}
		\item{$\alpha\vec{u}$}
		\item{$\vec{u}=0$ as shown above, as it is a solution}
		\item{$\alpha 0=0$}
	\end{enumerate}
	\item{The solution of the previous did form a subspace}
\end{enumerate}
\par{Consider the next linear equation $x+2y-3z=10$. This fails the first test of not containing the origin/0 vector. Even if we try to test for closure under addition, we get that that test fails too. This is because the right hand side of the equation is non-zero, i.e: there is a constant within the linear equation, i.e, the solution is non homogeneous. The solutions to a homogeneous system of linear equations \textbf{always forms a subspace}. The inverse of that is also true, in that \textbf{the solutions to a non-homogeneous set of linear equations is never a subspace}.}
\subsection{What are all the subspaces of $\mathbb{R}^{2}$?}
\begin{itemize}
	\item{$S=\lbrace\left(0,0\right)\rbrace$ is a subspace, as it contains the 0 vector and is closed both under addition and multiplication.}
	\item{Any straight line going through the origin is a subspace, as it contains the origin by design. If you added any 2 points on the line, as the gradient remains the same, it will just end up somewhere else on the line. This is the same for a scalar multiple}
	\item{$\mathbb{R}^{2}$ Itself, given enough points added/large enough scalar, we can 'tile' the whole of $\mathbb{R^{2}}$, therefore itself must be a subspace.}
\end{itemize}
\subsubsection{What are all the subspaces of $\mathbb{R}^{3}$?}
\par{A similar argument is made for $\mathbb{R}^{3}$ as for $\mathbb{R}^{2}$}
\begin{itemize}
	\item{The origin itself: $S=\lbrace(\left(0,0,0\right)\rbrace)$}
	\item{Planes through the origin}
	\item{Lines through the origin}
	\item{The whole of $\mathbb{R}^{3}$}
\end{itemize}
\subsection{Linear combinations}
\par{A linear combination of $v$ and $w$ is any vector in the form: 
	\[
	\alpha v+\beta w
	\] 
Where $\alpha,\beta\in\mathbb{R}$}
\section{Span and Spanning set}
\par{Given: $T=\lbrace(v_{1},v_{2},v_{3}...v_{k}\rbrace)$, the \textbf{span} of $T$ is:
\[
	\text{span($T$)}=\lbrace\left(\alpha_{1}v_{1}+\alpha_{2}v_{2}+\alpha_{3}v_{3}...\alpha_{k}v_{k}\right): a_{i}\in\mathbb{R}\rbrace
\] 
The span of $T$ is exactly the set of all linear combinations that can be formed from the vectors in $T$, which forms a subspace (The span of anything is a subspace, and the Span of T is the smallest subspace containing T). 

The span of an empty set is defined to be 0: span$\left(\emptyset\right)=0$.
}
\subsection{Testing For Membership Of Span And The Spanning Set}
\par{To test for membership of 1 vector in a span, we simply use a set of linear equations. E.g:}
Is: $\left(1,0,1\right)\in\text{span}\left(\lbrace\left(1,2,3\right),\left(2,3,4\right)\rbrace\right)$ ?
$$\alpha\left(1,2,3\right)=\beta\left(2,3,4\right)=\left(1,0,1\right)$$
$$
\begin{pmatrix}
	\alpha +2\beta=1 \\
	2\alpha+3\beta=0 \\
	3\alpha+4\beta=1 \\
\end{pmatrix}
$$
Remember the span returns the full set, whereas the spanning set refers to the set which builds the span.
\subsection{Verifying a Spanning Set}
\par{Given a subspace $S$, how do we prove that $T$ is a spanning set? We must verify that all vectors in the span can be made from linear combinations of the spanning sets, as well as the spanning sets being unable to make anything outside of the span. E.g:

	Show that span$\left(\left(1,1,0\right), \left(0,0,1\right)\right)=\lbrace(\left(x,x,z\right):x,z\in\mathbb{R}\rbrace)$
}
\begin{enumerate}
	\item{What vectors are in: span$\left( \left(1,1,0\right), \left(0,0,1\right)\right)=\lbrace( \left(x,x,z\right) :x,z\in\mathbb{R} \rbrace)$}
	\item{All vectors of the form: $\alpha\left(1,1,0\right)+\beta\left(0,0,1\right)=\left(\alpha,\alpha,\beta\right)$}
	\item{This is similar to the above answer of $\left(x,x,z\right)$, so it is the same.}
\end{enumerate}
\subsection{Finding a small spanning set for $S$}
\begin{enumerate}
	\item{Algorithm: choose some vector $v_{1}\in S$ and examine span$\left(v_{1}\right)$}
	\item{Check if this is enough to fill $S$. If it isn't enough, there are some vectors that you can span, and some which you can't.}	
	\item{Choose a $v_{2}\in S$ to try and reach all the vectors that $v_{1}$ cannot reach.}
	\item{Keep going until S can be contained}
\end{enumerate}
\par{If $S$ is given in a \textbf{parametric form} then there is an easy way to find the spanning set.}
\begin{align*}
	& S=\lbrace\left(x,2x,z,-z\right): x,z,\in \mathbb{R}\rbrace \\
	& \vec{v_{1}}=\left(1,2,0,0\right) \\
	& \vec{v_{2}}=\left(0,0,1,-1\right) \\
	& \left(x,2x,z,-z\right)=\left(x,2x,0,0\right)+\left(0,0,z,-z\right)
\end{align*}
\section{Linear Independence}
\par{A set of vectors is linearly dependent if one of them is a linear combination of the others. E.g: \[
\lbrace\vec{v_{1}}, 2\vec{v_{1}}\rbrace
\] We see here that the 2nd coordinate is just 2 times the first coordinate. A set of size 2 is linearly dependent only when one is a multiple of the other.}
\subsection{Linear Independence Test}
\par{in $\mathbb{R}^{n}$, a set $T=\lbrace v_{1}, v_{2}, v_{3}...v_{k}\rbrace$ is linearly independent if and only if the homogeneous system of n linear equations in the unknowns has a unique solution. E.g: }
\begin{align*}
	& T=\lbrace\left(1,0,0\right), \left(0,1,1\right), \left(1,1,2\right)\rbrace \\
	& \text{Solve } \\ 
	& \alpha_{1}\left(1,0,0\right)+\alpha_{2}\left(0,1,1\right)+\alpha_{3}\left(1,1,2\right)=\left(0,0,0\right)
\end{align*}
The solution must always equal $\left(0,0,0\right)$ as it should be homogeneous
\subsection{Properties of linear independence}
\begin{itemize}
	\item{A subset of a linearly independent set is linearly independent.}
	\item{A superset of a linearly dependent set is dependent.}
\end{itemize}
\par{This is quite common sense as if you have a dependent set, adding something extra won't change it to be independent. Another idea is that the span would not change if a dependent vector is removed from the set of vectors. 

Suppose we have a set 'S' with multiple vectors within, but $v_{1}$ is linearly dependent on a linear combination of the other vectors: \[
S=\lbrace v_{1}, v_{2}, v_{3}... v_{k}\rbrace 
\] \par{Then, as $v_{1}$ is a linear combination of some other vectors: \[
 v_{1}=\alpha_{1}v_{1}+...\alpha_{k}v_{k}
\] We can see that by removing $v_{1}$ the span of S will not change, as whatever $v_{1}$ can make can already be made by the vectors which make up $v_{1}$: \[
\mathrm{span}\left(S\right)=\mathrm{span}\left(S\backslash\vec{v_{1}}\right)
\] }
\begin{itemize}
	\item{Linearly dependent set}
		\begin{itemize}
			\item{Set has \textbf{redundancy}}
			\item{Can remove dependent vectors without changing the span (as explained above}
		\end{itemize}
	\item{Linearly independent set}
		\begin{itemize}
			\item{No redundancy}
			\item{every vector is needed}
		\end{itemize}
\end{itemize}
\section{Basis}
\par{Let $S$ be a subspace of $\mathbb{R}^{n}$, then a \textbf{basis} for $S$ is a set $B$ of vectors in $S$ such that 
\begin{itemize}
	\item{$S=\mathrm{span}\left(B\right)$}
	\item{$B$ is linearly independent}
\end{itemize}
B shall contain exactly the vectors you need to \textbf{span} $S$, or to create every element within $S$. A basis is the \textbf{most efficient way} to specify a subspace. Given a basis, you can determine which vectors are in the span of that basis or not. }
\subsubsection{The Standard Basis}
\par{The standard basis is the most simple basis for $\mathbb{R}^{n}$, given by:}
\begin{align*}
	& e_{1}=\left(1,0,0,....,0\right) \\
	& e_{2}=\left(0,1,0,....,0\right) \\
	& e_{3}=\left(0,0,1,....,0\right) \\
	& e_{n}=\left(0,0,0,....,1\right) \\
\end{align*}
\subsection{Dimension}
\par{Every basis for a subspace $S$ has the  \textbf{same number} of vectors. This number is called the dimension of $S$, denoted simply as  $\mathrm{dim}\left(S\right)$. E.g: anything spanned by 1 vector is an one-dimensional object, everything spanned by 2 vectors is a 2-dimensional object.

A basis for a subspace satisfies the "Goldilocks property".
\begin{itemize}
	\item{It is only \textbf{just big enough} to be a spanning set}
	\item{it is only \textbf{just small enough} to be independent}
\end{itemize}
Let $S$ be a $k$-dimensional subspace of $\mathbb{R}^{n}$. Then:
\begin{itemize}
	\item{Any set of \textbf{less than} k is not a spanning set of $S$}
	\item{Any spanning set for $S$  \textbf{contains} a basis for $S$}
	\item{Any spanning set for $S$ of size \textbf{exactly} $k$ is a basis for $S$}
\end{itemize}
}
\subsubsection{Example}
\par{Give a basis for the solution space of the SLE: }
\begin{align*}
	& a+b+2c+d-e=0 \\
	& 2a+2b+5c+d-2e=0 \\
\end{align*}
\[
	\left[
	\begin{array}{ccccc|c}
		1 & 1 & 2 & 1 & -1 & 0 \\
		2 & 2 & 5 & 1 & -2 & 0
	\end{array}
	\right]
\] 
\par{A single row operation gives the following: $R2\leftarrow R2-2R1$}
\[
	\left[
		\begin{array}{ccccc|c}
			1 & 1 & 2 & 1 & -1 & 0 \\
			0 & 0 & 1 & -1 & 0 & 0
		\end{array}
	\right]
\] 
\par{From the above, we see a and c (first and 3rd column) are basic variables, whereas b, d and e are non-basic/free variables. We now do back substitution where :
\begin{itemize}
	\item{e is a parameter}
	\item{d is a parameter}
	\item{c is a basic variable, which from the 2nd row can be expressed as: $c-d=0$, therefore  $c=d$}
	\item{a is basic, so from the first row, we get $a+b+2c+d-e=0$, as $c=d$, we get  $a=e-b-3d$}
\end{itemize}
Every variable is now expressed as a free variable. There are 3 free variables, so the solution space is 3 dimensional. $S=\lbrace\left(e-b-3d, b, d, d, e\right)|b,d,e\in\mathbb{R}\rbrace$}. The parametric form gives us the basis in the following: 
\begin{align*}
& S=\lbrace\left(e-b-3d, b, d, d, e\right)|b,d,e\in\mathbb{R}\rbrace \\
& =b\left(-1,1,0,0,0\right)+d\left(-3,0,1,1,0\right)+e\left(1,0,0,0,1\right)
\end{align*}
This is because for b, the first term contains a negative b, and the 2nd term a positive b, for d, the first term contains a negative 3, and the 3rd and 4th terms, a positive d, and finally for e: the first and last terms contain an e. Because of back substitution, they are forced to be linearly independent, they span the whole set, so they are the basis for the solution space: \[
B =\lbrace\left(-1,1,0,0,0\right),\left(-3,0,1,1,0\right),\left(1,0,0,0,1\right)\rbrace
\]
\chapter{Matrix Algebra}
\par{Matricies can be added, multiplied by other matrices or by scalars or transposed. }
\subsection{Transposition}
\par{To transpose a matrix (denoted by the operation $\left(AB\right)^{T}$, we take each row and write it as its corresponding column}
\subsection{Addition}
\par{Matricies can be added simply like vector addition}
\subsection{Matrix Multiplication}
\section{Basic square Matricies}
\par{There are 2 square matrices that are defined}
\subsection{The 'zero' matrix}
\par{The zero matrix is just any matrix made up of only '0'. It is denoted by $O_{n}$, where no is the number of rows and columns}
\subsection{The 'identity' matrix}
\par{The identity matrix is defined as a square matrix of a 1 in the diagonal line: (identity matrix of size 3, or $I_{n}$)}
\[
	\left[
		\begin{matrix}
			1 & 0 & 0 \\
			0 & 1 & 0 \\
			0 & 0 & 1 
		\end{matrix}
	\right]
\] 
\par{Reminder that the identity matrix acts as a '1' in the matrix theory, whereas the zero-matrix acts as a zero. This means that identity matrices' multiplication are commutative: }
\[
A\cdot I=I\cdot A=A
\] Where A is any matrix and I is the identity matrix of the same order \[
A+O=O+A=A
\] 
\subsection{Special Square Matricies}
\begin{itemize}
	\item{A matrix is \textbf{symmetric} if $A=A^{T}$}
	\item{A matrix is \textbf{skew-symmetric} if $A=-A^{T}$}
	\item{A matrix is \textbf{upper-triangular} if $A_{ij}=0$ for $i>j$ \[
	\left[
	\begin{matrix}
		x & y & z \\
		0 & a & b \\
		0 & 0 & c
	\end{matrix} 
	\right]
	\] }
	\item{A matrix is \textbf{lower-triangular} if $A_{ij}=0$ for $i<j$}
	\item{A matrix is \textbf{diagonal} if $A_{ij}$ for all $i\ne j$}
\end{itemize}
\subsection{SLE and matrices}
\par{We have $Ax=b$, where a is the coefficients, x is the variables and b is the values to the right of the augmentation bar. Let A be an $m\times n$, m rows n columns matrix. 
\begin{itemize}
	\item{The \textbf{Row Space of A} is the span of the rows, the set of all linear combinations of all the rows. It is a subspace of $\mathrm{R}^{n}$.}
	\item{The \textbf{Column Space of A} is the set of all linear combinations of its columns, the span of the columns. They are solved similarly to the row space, simply transpose prior to solving and then transpose back to obtain the answer. It is a subspace of $\mathrm{R}^{m}$}
	\item{The \textbf{Null Space of A} is the set of all solutions to $Ax=0$, it is a subspace of $\mathrm{R}^{n}$. To find the null space, you just have to solve the system of linear equations.}
\end{itemize}
}
\subsubsection{Dimensions}
\par{The row rank of A is the dimension of its row space. The column rank of A is the dimension of its column space and the nullity of A is the dimension of its null space. \[
\begin{matrix}
	1 & 2 & -1 & 3 \\
	1 & 0 & 3 & -1
\end{matrix}
\] The \textbf{row rank} is 2. The \textbf{column rank} is the whole of 2. THe row rank is always equal to the column rank. The \textbf{Nullity} is given by: \[
\left[
	\begin{matrix}{cccc|c}
	1 & 2 & -1 & 3 & 0\\
	1 & 0 & 3 & -1 & 0
\end{matrix}
\right]
\] \[
\left[
\begin{matrix}{cccc|c}
	1 & 2 & -1 & 3 & 0 \\
	0 & -2 & 4 & -4 & 0
\end{matrix}
\right]
\] }
\subsubsection{Example}
\par{We have a $3\times4$ matrix: } 
\[
\left[
	\begin{matrix}
		1 & 0 & 1 & 1 \\
		1 & 1 & 3 & 2 \\
		2 & 1 & 4 & 3 
	\end{matrix}
\right]
\]
\par{The third row is dependent on the first 2, so everything built with the 3rd row can be replaced by a combination of 1 and 2. Therefore the basis is just the first and the 2nd row: \[
B=\lbrace\left(1,0,1,1\right),\left(1,1,3,2\right)\rbrace
\] it is important to know that the dimension is equal to the number of free variables (think about it, it is the dimension that can be made. ) }
\subsubsection{conclusion}
\par{The rank is the dimension of the row-space or column-space and the nullity is the dimension of the solution space. The dimension is the number of free variables in that subspace.}
\subsubsection{Rank-Nullity THeorem}
\par{if $A$ is  $m\times n$ then  \[
\mathrm{rank}\left(A\right)+\mathrm{nullity}\left(A\right)=n
\] }
\subsubsection{Why?}
\par{The reason is that each non-zero row gives us a leading entry, leading to the number of basic variables. Which is the rank. The Non-basic variables are the free variables, which give us the nullity. Each variable either contributes to the rank, or the nullity.}
\subsubsection{Other properties of nullity and rank}
\par{Recall that a homogeneous SLE is one where the matrix of coefficients $A$ and the column of solution variables vector  $B$ multiply to give  0.  $Ax=0$. As a result, a non-homogeneous SLE must be one where $Ax=b,b\ne0$. Also remember that given the solutions x and y, we can add the solutions together to obtain another solution. }
\begin{align*}
	Ax&=0 \\
	Ay&=0 \\
	A\left(x+y\right)&=0 
\end{align*}
\par{From this, we concluded that the solutions of a homogeneous SLE formed a subspace. 

	If we think about our non-homogeneous SLE $Ax=b$ again, and we think about what would happen when 2 solutions subtract from each other then their difference will be a solution to the homogeneous version of the solution.

	This implies that when given a solution space (the nullspace) to a homogeneous SLE, any solution to non-homogeneous SLEs will just be a translate of that nullspace.
}

\subsection{Inverses}
\par{If $A$ is an  $n\times n$ matrix and \[
AB=I_{n}
\] then \[
BA=I_{n}
\] and we call $B$ the inverse of  $A$ and that  $A$ is invertible. Note: $I_{n}$ is the identity matrix. An example: the zero-matrix is non-invertible as anything you multiply it by will never give the identity matrix.}
\subsubsection{Finding the inverse}
\par{To find the inverse of an $n\times n$ matrix of  $A$:
\begin{enumerate}
	\item{Form the super-augmented matrix}
	\item{Performe Gauss-Jordan elimination to try and reach reduced row echelon form \[
			\left[I_{n}|X\right]
	\] }
	\item{If successful, then $X=A^{-1}$ and if not successful, $A$ is not invertible.}
\end{enumerate}
An $n\times n$ is invertible  \textbf{if and only if} its reduced row echelon form is $I_{n}$ This occurs if and only if the rank is equal to $n$. A can only be invertible if the columns and rows are both linearly independent. This also means that for $n
times n$ matrices, it is only invertible if the rank is $n$. From this we can see that either all the following conditions are true, or none are: 
\begin{itemize}
	\item{A is invertible}
	\item{A has full rank (i.e, rank=maximum number of rows=maximum number of columns)}
	\item{The rows of A are linearly independent}
	\item{The columns of A are linearly independent}
\end{itemize}
Also note that, if $A, B$ are both invertible, then:
\begin{itemize}
	\item{$AB$ is invertible}
	\item{$A^{2}$ is invertible}
	\item{$A^{T}$ is invertible}	
\end{itemize}
}
\subsection{Determinants}
\par{A determinant is essentially just a number associated with any square matrix, given by: \[
\mathrm{det}\left(\left[		
	\begin{matrix}
			a & b \\
			c & d
	\end{matrix}
\right]\right)=ad-bc
\] You can also use this vertical bar notation: \[
\left|
	\begin{matrix}
			a & b \\
			c & d
	\end{matrix}
\right|=ad-bc
\] It is also somehow related to inverses: \[
\left[		
	\begin{matrix}
			a & b \\
			c & d
	\end{matrix}
\right]^{-1}=\frac{1}{ad-bc}\left[
	\begin{matrix}
			d & -b \\
			-c & a
	\end{matrix}
\right]
\] or \[
\left[		
	\begin{matrix}
			a & b \\
			c & d
	\end{matrix}
\right]^{-1}=\frac{1}{\mathrm{det}\left(\left[
\begin{matrix}
			a & b \\
			c & d
\end{matrix}
\right]\right)}\left[
	\begin{matrix}
			d & -b \\
			-c & a
	\end{matrix}
\right]
\]  
}
\subsubsection{$3\times 3$ matrix determinants}
\par{3 by 3 matrices are defined recursively due to \textbf{cofactors}. We select any row, what ever row. And then we multiply by 1,-1,1,-1 until we reach the end of the row. E.g: \[
\mathrm{det}\left[
	\begin{matrix}
		2 & 1 & -1 \\
		0 & 1 & 1 \\
		1 & 1 & -1 
	\end{matrix}
\right]
\] we multiply as such: $2 \times 1$,  $1 \times -1$,  $-1 \times 1$ and we get:  $2,-1,-1$. Then for 2, since it is the first column, we consider the 2nd column, for -1, since it is the 2nd column, we consider the first and third, and finally for the third column, the first 2 columns. And such the determinant can be valuated as follows: \[
2\left|
\begin{matrix}
	1 & 1\\ 1 & -1
\end{matrix}
\right| -1\left|
\begin{matrix}
	0 & 1\\ 1 & -1
\end{matrix}
\right|-1\left|
\begin{matrix}
	0 & 1\\ 1 & 1
\end{matrix}
\right|=-2
\] We would get the same answer no matter which row/column we do the expanding of cofactors. This gets very problematic with larger matrices, as for $4\times 4$ matrices, we need to solve 4  $3\times 3$ matrices, and for that we need 3 $2\times 2$ matrices and as you can see, with larger matrices this would be very tedious. 
}
\subsubsection{Some properties of determinants}
\par{Notice that the determinant of the transpose is equal to the determinant of the non-transpose \[
\mathrm{det}\left(A^{T}\right)=\mathrm{det}\left(A\right)
\] if $A$ has a row of zeroes, then  $\mathrm{det}\left(A\right)=0$. If however, if a matrix were to be upper triangular (meaning that everything below the diagonal is 0, then its determinant has to be the product of the entries on the diagonal) $\left(a_{11}\times a_{22}\times a_{33}...a_{nn}\right)$. By using the previous 2 properties, we can use row reduction/Gaussian elimination to obtain a form that is easier to solve (be it 1 row of zeros, or upper triangular). Using Gaussian elimination does change the matrix, but in predictable ways, allowing us to change the determinant back to what it should be.}
\subsubsection{Types of changes}
\par{Following from above, here are a list of the changes. 
\begin{itemize}
	\item{$\mathrm{det}\left(A'\right)=-\mathrm{det}\left(A\right)$ if the ERO is Type 1: $R_{i}\leftrightarrow R_{j}$}
	\item{$\mathrm{det}\left(A'\right)=\alpha\mathrm{det}\left(A\right)$ if the ERO is of Type 2: $R_{i}\leftarrow\alpha R_{i}$}
	\item{$\mathrm{det}\left(A'\right)=\mathrm{det}\left(A\right)$ if the ERO is of Type 3: $R_{i}\leftarrow R_{i}+\alpha R_{j}$}
\end{itemize}}
\subsubsection{SOme more properties of determinants}
\par{Know that \[
\mathrm{det}\left(\alpha A\right)=\alpha^{n}\mathrm{det}\left(A\right)
\] Where n is the rank, also that A is only invertible if the determinant of A is not equal to 0.

The determinant is also multiplicative, meaning that: \[
\mathrm{det}\left(AB\right)=\mathrm{det}\left(A\right)\times\mathrm{det}\left(B\right)
\] as a result: \[
\mathrm{det}\left(AB\right)=\mathrm{det}\left(BA\right)
\] and \[
\mathrm{det}\left(A^{k}\right)=\left(\mathrm{det}\left(A\right)\right)^{k}
\] and if A is invertible (note $A^{-1}$ is inverse, not to the -1 power): \[
\mathrm{det}\left(A^{-1}\right)=\frac{1}{\left(\mathrm{det}\left(A\right)\right)}
\] 
}
\section{Function Terminology}
\par{A function has a \textbf{domain} $A$ and a \textbf{co-domain} $B$, and maps each $a\in A$ to  \textbf{exactly one} element in $B$. Unique input to unique output, or... Different inputs may have the same output, but 1 input cannot be mapped to 2 different outputs. The notation is: \[
f\left(\left(x,y\right)\right)=\left(expr1, expr2\right)
\] A linear transformation is a function $f:\mathbb{R^{n}}to\mathbb{R^{m}}$ such that: \[
f\left(u+v\right)=f\left(u\right)=f\left(v\right)
\] For all vectors $u,v\int\mathbb{R^{n}}$ 
and:
\[
f\left(\alpha u\right)=\alpha f\left(u\right)
\] for all vectors $u\int\mathbb{R^{2}}$ and scalar $\alpha\in\mathbb{R}$

Given a function $f$, decide whether or not it is a linear transformation. If it is, there has to be a symbolic proof, but if it isn't, 1 counterexample is enough.
Example of not a linear transformation: \[
g:\mathbb{R^{2}}\to\mathbb{R^{2}}:\left(x,y\right)\to\left(x^{2},y^{2}\right)
\] scaling it fails immediately: \[
g\left(2\left(1,1\right)\right)\ne2g\left(\left(1,1\right)\right)
\] This is a similar proof to subspaces. 
As a result of the above, there exists this property that if $f:\mathbb{R^{n}}\to\mathbb{R^{m}}$ is a linear transformation then \[
f\left(0\right)=0
\] If the formulas has any constants, it immediately is obvious that it is non-linear as it violates the above (will not get 0 if we put in 0).
}
\subsection{Kernel}
\par{If $f:\mathbb{R^{n}}\to\mathbb{R}^{m}$, then the \textbf{kernel} of $f$ is the vectors in the  \textbf{domain} $\mathbb{R^{n}}$ that are mapped to 0. It is sometimes know just as $K$. The kernel cannot be empty, containing at least the $\vec{0}$ (zero vector) but can contain more. The kernel itself is also \textbf{always} a subspace. An example: \[
f\left(x,y\right)=\left(2x-y,2y-4x\right)
\] Find the kernel of the above:
\begin{itemize}
	\item{The 0 vector is always in the kernel, and as there are no constants, it is confirmed in the kernel} \\
	\item{$\left(1,2\right)$ via substitution}
	\item{drawing a line from the 2 above points shows that every point on that line is mapped to 0, and everything not on that line does not map to 0. As a result the kernel of $f$ is a subspace, is a line.}
\end{itemize}

Proving that the kernel is a subspace:
\begin{enumerate}
	\item{Is $0\in \mathrm{kernal}\left(f\right)$, yes, because  $f\left(0\right)=0$}
	\item{Is it closed under addition?
		\begin{enumerate}
			\item{Let $\vec{u},\vec{v}\in \mathrm{kernal}\left(f\right)$}
			\item{is $\vec{u}+\vec{v}\in \mathrm{kernal}f$?\[
			f\left(\vec{u}+\vec{v}\right)=f\left(\vec{u}\right)+f\left(\vec{v}\right)=0+0=0
			\] }
		\end{enumerate}}
	\item{Is it closed under multiplication? Yes because anything times 0 is 0.}
\end{enumerate}}
\subsection{Range}
\par{If $f:\mathrm{R^{n}}\to\mathrm{R^{m}}$, then the \textbf{range} of $f$ is the set of vectors:  \[
\lbrace f\left(u\right):u\in\mathrm{R^{n}}\rbrace
\] This is a set of vectors in the \textbf{co-domain} of $f$, namely $\mathbb{R}^{m}$.

As 0 (domain) is always mapped to 0 (range), we know 0 will always be in the range.
}
\subsection{Determining linear transformations}
\par{A linear transformation is \textbf{determined} by its action on a \textbf{basis} of the domain. Suppose that $f:\mathbb{R}^{2}\to \mathbb{R}^{3}$ is a linear transformation such that: \[
		f\left(1,0\right)=\left(1,1,1\right) \text{ and } f\left(0,1\right)=\left(0,0,1\right)
\] then we can obtain: \[
f\left(1,1\right)
\] through: \[
f\left(\left(1,0\right)+\left(0,1\right)\right)
\] Think of it as a linear combination, where the scalars are given in the $\mathbb{R}^{2}$ and the vectors are $\mathbb{R}^{3}$ 

As a result, the range of a linear transformation $f:\mathbb{R}^{n}\to\mathbb{R}^{m}$ is the span of the images of any basis of $\mathbb{R}^{n}$

The range and kernel of a linear transformations are subspaces.

To find the basis for a kernel, we must solve $f\left(v\right)=0$
}


\end{document}

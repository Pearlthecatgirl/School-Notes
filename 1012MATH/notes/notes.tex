\documentclass[11pt]{book}
\usepackage[margin=2cm]{geometry}
\usepackage{amsmath}
\usepackage{amssymb}
\author{pearl}
\title{MATH1012 Notes}

\begin{document}
\maketitle
\chapter{Linear Algebra}
\section{Set of linear equations}
\subsection{Definitions}
\begin{itemize}
	\item{Linear equation: is an equation whose terms are numbers, and variables to the first power e.g $x+y=3$ is linear, but  $x^{2}+4y=5$ isn't.}
	\item{Solution: a solution is a choice of values (numbers) that satisfies a linear equation.}
	\item{System of linear Equation: is just a collection of linear equation. A solution to a system linear equation is a set of solutions which satisfy all equations. Give solution using set notation. E.g: $S=\{\left(2,1\right)\}$ for the graph: $x+2y=4, x-y=1$ an SLE in 2 dimensional space can either have:
	\begin{itemize}
		\item{No solutions (inconsistent): e.g 2 parallel lines}
		\item{unique solution: 2 intersecting lines}
		\item{Infinitely many solutions: 2 lines are the same line/overlay-ed on top of each other}
	\end{itemize}
	}
\end{itemize}
\subsubsection{Solving set of linear equations}
\par{Solve them geometrically or solve them through algebra. Example below:}
\begin{align*}
	x+y+z=1 \\ 
	x+y-z=1
\end{align*}
\subsubsection{Solving geometrically}
\begin{enumerate}
	\item{guess 1 point from each equation: (1,0,0) and (0,1,0)}
	\item{The line that connects up the points is a solution}
\end{enumerate}
\subsubsection{Solving algebraically}
\begin{enumerate}
	\item{Subtraction method. 'Subtract' one equation from the other
		\begin{align*}
			(x+y+z=1) - (x+y-z=1)\\
			\implies 0 + 0 + 2z=0 \\
			\implies z=0 \\
			\implies x+y+0=1 \\
			\implies x+y=1
		\end{align*}
		}
\end{enumerate}
\subsection{Parametric Solutions}
\par{In a parametric solution, every coordinate is either \textbf{constant, free variable, linear combination }(constants and free variables). i.e expressing the equation in terms of one variable. We can take x as the free variable, or y. Using the above as an example:}
\begin{align*}
	x+y+z=1 \\ 
	x+y-z=1 \\
	\implies x+y=1 \\
	\implies x=1-y \\
	\implies y=1-x \\
	\implies z=0 \\
	\therefore S=\{\left(x, 1-x, 0\right)\ \|x\in \mathbb{R} \}
\end{align*}
\par{Some variables cannot be chosen as free variables, in this example: z.}
\subsection{Elementary Row Operations}
\par{Given a set of linear equations: 
\begin{align*}
	x+y=5 \\
	2x-2y=7
\end{align*}
We can do the following and have no effect on the solution set:
\begin{itemize}
	\item{swap 2 equations' locations}
		\begin{align*}
			2x-2y=7 \\
			x+y=5
		\end{align*}
	\item{Multiply an equation by a non-zero number}
		\begin{align*}
			x+y=5 \\
			\text{becomes} \\
			2x+2y=2 \times 5 \\
			= 2x +2y=10
		\end{align*}
	\item{Add a multiple of one equation to another}
		\begin{align*}
			x+y=6 \\
			2x+2y=10 \\
		\end{align*}
		$R2\leftarrow R2-R1$
		\begin{align*}
			x+y=6 \\
			-4y=-3 \\
			y=\frac{-3}{-4} \\
			y=\frac{3}{4}
		\end{align*}
\end{itemize}
}
\subsection{Matrix notation and Gaussian elimination}
\par{Systems of linear equations can be expressed as a matrix, where the variables (x,y,z , etc) are omitted and only the coefficients are written down. Example:
\begin{align}
	x+2y=7 \\
	2x-y=4
\end{align}
Becomes:
\[
\left[
\begin{array}{cc|c}
	\text{x column} & \text{y column} & \text{solution} \\
	1 & 2 & 7 \\
	2 & -1 & 4
\end{array}
\right]
\]
\subsection{Row Echelon Form}
\par{A matrix is in row echelon form if:
	\begin{itemize}
		\item{The rows that are \textbf{all-zero} should be at the bottom}
		\item{The \textbf{leading entry} of every other row is \textbf{further to the right} than the leading entry of any previous rows}
	\end{itemize}
}

\end{document}




%--Preamble here --%
\documentclass[11pt]{report}
\usepackage[a4paper, total={6in,8in}]{geometry}
\usepackage{amsmath}
\usepackage{tikz}
\title{Math Notes}
\author{Pearl Lin}
%--Document here--%
\begin{document}
\maketitle
\tableofcontents

\chapter{System Of Linear Equations}
\paragraph{A system of linear equations is literally a set of linear lines. The number of total terms correspond to the number of dimensions. For example:}
$$
\begin{bmatrix}
	ax+by+cz=n_{1} \\
	dx+ey+fz=n_{2} \\
	gx+hy+iz=n_{3} \\
\end{bmatrix}
$$
\paragraph{The above corresponds to a 3 dimensional space with 3 planes (2d would be a line) where $a\to i$ and $n_{1\to3}$ correspond to constants and x,y,z correspond to the 3 units for space}

\section{Row Echelon Form}	
Row Echelon form/row reduction is a way to solve equations that do not fall into the category of 0 connections and every connection. You start in Row Echelon Form and through Gaussian Elimination/Row Reduction, \textit{Reduced Row Reduction Form} is achieved. 

\subsection{Row Operations}
There are 3 main operations:
\begin{enumerate}
	\item{$R_{i}\leftrightarrow R_{2}$: Exchange equations between $R_{i}$ and $R_{j}$}
	\item{$R_{i}\leftarrow aR_{j}$: Multiply $R_{j}$ by constant $a$ and save into $R_{i}$}
	\item{$R_{i}\leftarrow R_{i}+R_{j}$: Sum $R_{i}$ and $R_{j}$ and save into $R_{i}$}
\end{enumerate}
The 3 operations are used to achieve \textit{Reduced Row Reduction Form}.

\end{document}

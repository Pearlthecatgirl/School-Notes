\documentclass[10pt]{report}

%preamble stuff here :3
\usepackage{amsmath}
\usepackage[margin=2cm,nohead]{geometry}
\author{Pearl Lin 24465993}
\title{Phys Notes}

\begin{document}
\maketitle
\chapter{Special Relativity}
\section{Definitions}
\begin{itemize}
	\item{Observer: An observer is an individual armed with measuring equipment like rulers and timers}
	\item{Frame of reference: An object/body or collection of bodies that is considered to be at rest}
	\item{Coordinate system: a "choice" by an observer on how to uniquely label points in space/time in their Frame of Reference}
	\item{Inertial Frame of reference: a frame of reference in which Newton's 1st law holds (inertial means not accelerating! Non inertial means accelerating)(does not accelerate or rotate)}
	\item{inertial observer: an observer that is within an inertial frame of reference}
	\item{inertial coordinate system: a coordinate system within an inertial frame of reference which uses:}
		\begin{itemize}
			\item{an ideal clock for measuring the time}
			\item{a Cartesian (x,y,z) system for locating points in space where things happen}
		\end{itemize}
	\item{an event: something which happens at a particular point within space/particular point in time}
	\item{Space/time coordinate: An inertial observer using an inertial coordinate system (using (x,y,z) Cartesian system and T for time) will assign to every event a single space time coordinate (t,x,y,z)\\ It is also the collection of all possible events, i.e, every (t, x, y, z) is called space-time (Minnoski space)}
	\item{Temporal separation (time interval) between events 1 and 2 is given by: \[
	\Delta t=t_{2}-t_{1}
	\] 
	}
	\item{Distance between events 1 and 2 is given by:
\begin{align*}
	\sqrt{\left(x_{2}-x_{1}\right)^{2}+\left(y_{2}-y_{1}\right)^{2}+\left(z_{2}-z_{1}\right)^{2}} \\=\sqrt{\Delta x^{2}+\Delta y^{2}+\Delta z^{2}}
\end{align*}
	}
	\item{Proper Time: Is the time duration given by the observer who sees two events take place at the same location in space: E.g If I get in an elevator, I see the building move up and down, and they stop/start at the same location.}
	\item{Proper Length: Is the length of an object given by an observer who sees the object as being still/at rest.}
\end{itemize}
\section{The Principle of Relativity}
\begin{itemize}
	\item{The laws of physics are the same in all inertial reference frames}
	\item{The outcome of any experiment is the same when performed with an identical initial conditions relative to any inertial frame of reference}
\end{itemize}
\subsection{Postulates of \textbf{Special Relativity}}
\par{The 2 postulates of \textbf{Special Relativity} are as follows:
\begin{itemize}
	\item{Postulate 1: The principle of relativity: The laws of physics are the same in all \textbf{non inertial frames of reference}}
	\item{Postulate 2: The speed of light: In a vacuum, in all inertial frames, light propagates in a straight line at the same speed, denoted \textit{c} at all times and in all directions. The speed of light is invariant.}
\end{itemize}
\subsection{Relative Quantities/Invariant Quantities}
\par{A relative quantity is a physical quantity which different observers may measure to have different values (e.g I am on a train with a ball moving at constant velocity. I observe that the ball is not moving (v=0) hence its kinetic energy is 0. Another observer outside of the train observes the ball to be moving, hence a non-zero kinetic energy.) In contrast, invariant quantities have the same value for all observers (e.g, mass)}
\subsubsection{Simultaneity is relative}
\par{Let observer 'A' be in a spaceship, there is a light-bulb in the middle of the spaceship, shooting 2 beams: one to the front and one to the back. From 'a''s frame of reference, the 2 lights will appear to hit the wall at the same time. Another observer 'b', who is observing in the middle of space, sees that the beam towards the back of the space ship will hit the wall before the beam towards the front of the spaceship as the spaceship travels forward. This is relative simultaneity. It is important to note that if 1 event is caused by another, then those events \textbf{MUST} be be observed in the same order for all observers. An example is that this instant must have occurred after the creation of \LaTeX, as this document is being written in \LaTeX. This fact will be observed in the same order for all observers.}
}
\subsubsection{Some Lengths stay the same}
\par{\textbf{Claim :} Lengths that are transverse to the direction of movement (at a right angle) do not change.\\\textbf{Proof by contradiction: Assume that a length that is transverse to the direction of movement changes with its motion} 2 Observers: 'a' and 'b'. Observer 'a' is on the train and 'b' is observing the train from the side. When the train is at rest, both agree on the height of the train to be 'h'. Ahead of the train, a tunnel with a clearance height of 'h' is observed. In the frame of reference of observer 'b', the tunnel is stationary and observer 'a' is moving towards it at a speed close to \textit{c} (speed of light). The assumption states that 'a' (in the same frame of reference as the train) will observe the height of the train to remain at 'h' while 'b' should in theory observe the height of the train to be some height less than h. This is a contradiction as the train cannot both fit and crash into the tunnel, meaning our assumption is false (and the claim is correct).
}
\subsubsection{Derivation of \textbf{'Time Dilation'} formula}
\par{Suppose there are 2 observers, 'a' and 'b'. Observer 'a' is currently on a train travelling at relativistic speeds within the frame of reference of observer 'b'. Observer 'a' places a photon gun towards the roof to shoot a beam of photons, and a mirror directly above the gun, so that it reflects the photon beam back down. The time taken for a singular photon packet to travel from the gun to the mirror and get reflected back down is given by the following: \[
t _{Train}=\frac{2d}{c}
\] 
Where:
\begin{itemize}
	\item{$t _{Train}$ is the time taken from the frame of reference on the train}
	\item{$d$ is the distance from the gun barrel to the roof}
	\item{$c$ is the speed of light}
\end{itemize}
This is easily derived as the total distance travelled by the photon packet is $2d$ (from the floor to the ceiling and from the ceiling to the floor) being substituted into  $v=\frac{s}{t}$ relationship between velocity, distance and time. From observer 'a''s perspective, the train is not moving (it is at constant velocity and hence, a non-inertial frame of reference). From observer 'b''s perspective, however, the train had since moved a certain distance $r$. The path the photon takes is now 2 times the hypotenuse of the triangle formed by $\frac{r}{2}$ and $d$ (height of the train) (as it is the path from the floor to the ceiling and the back down). When we compare the 2 times by joining them at the same variables and then rearrange, we get the following:
\[
t_{Observer}=\frac{t_{Train}}{\sqrt{1-\frac{v^{2}}{c^{2}}}}
\] 
We denote the following term:
\[
\gamma=\frac{1}{\sqrt{1-\frac{v^{2}}{c^{2}}}}
\] 
as the 'Lorentz' factor/gamma factor. Then, the time dilation formula simply becomes
\begin{align}
	t_{dilated}=t_{un-dilated}\times\gamma \\
	\gamma \ge 1
\end{align}
In general, the time dilation formula is:
\begin{align}
	\Delta t'=\gamma\Delta t_{0}
\end{align}
Where:
\begin{itemize}
	\item{$\Delta t_{0}$ is the time between 2 events as seen by an observer who sees events occur at the same time (proper time)}
	\item{$\Delta t'$ is the time between 2 events as seen by an observer who sees events occur at speed v relative to the first observer who measured the time (dilated time)}
\end{itemize}
\section{Coordinate Transformation}
\par{ 4 results established that collapse into the idea of coordinate transformation:
	\begin{itemize}
		\item{Simultaneity is relative}
		\item{lengths transverse to the direction of relative motion are unchanged}
		\item{Time dilation: moving clocks run slower (Moving entity experiences "less" time during the same trip)}
		\item{Length contraction: moving lengths shrink in the direction of motion}
	\end{itemize}
In general, a coordinate transformation is a set of mathematical expressions which relates any two sets of coordinate systems, whether they be the same or different frames of reference. }
\par{Consider 2 observers, S and S', each with their own inertial coordinate system:(t,x,y,z) and (t',x',y',z') respectively. Observer 'S''s coordinate system has its origin at the centre of the room with the z unit vector pointing straight up. S' has its origin in the same place, but S''s y coordinate goes in the opposite direction as S's x coordinate, and S''s x coordinate goes in the same direction as S's y coordinate. In short, x'=y, y'=-x, z'=z. That is the way to get from 1 coordinate to another.}
\subsubsection{Lorentz Transform}
\par{Consider 2 observers S and S'. Then the coordinate system for space-time used by both are given as follows: (t,x,y,z) and (t',x',y',z') respectively. The Lorentz transform is then used to change 1 set of coordinates into the other. When the following criteria are fulfilled, the arrangement is known as a \textbf{standard configuration}. While not necessary, the standard configuration may make dealing with frames easier:
	\begin{itemize}
		\item{The x, y, z coordinates of frame S are oriented parallel to that of S'}
		\item{Frame S' moves at a constant velocity (inertial frame)}
		\item{The origin coordinates in space coincide when S and S' have time=0}
	\end{itemize}
The Lorenz transformation is given by the following vector of operations:
\begin{align*}
	t'=\gamma\left(t-\frac{vx}{c^{2}}\right) \\
	x'=\gamma\left(x-vt\right) \\
	y'=y \\
	z'=z \\
	\text{where:} \\
	\gamma=\frac{1}{\sqrt{1-\frac{v^{2}}{c^{2}}}}
\end{align*}
(for movement across the x direction)
}
\section{Space-time intervals}
\par{If 2 observers moving relative to one another, and they both see the same 2 events, they will measure the time and distance between the 2 events differently, i.e: $\Delta t, \Delta x, \Delta y, \Delta z$ will be different between the 2 inertial observers (generally). However, lets first take the first observer to be S. Then  S will use the inertial coordinates: $\left(t,x,y,z\right)$. As a result, the time difference and space difference between the 2 events as observed by S is given as: $\left(\Delta t, \Delta x,\Delta y, \Delta z\right)$. The space-time interval $\left(\Delta S\right)^{2}$ between the 2 events is defined to be:  
\begin{align*}
\left(\Delta s\right)^{2}=c^{2}\left(\Delta t\right)^{2}-\left(\Delta x\right)^{2}-\left(\Delta y\right)^{2}-\left(\Delta z\right)^{2} \\
\implies\Delta s^{2}=c\Delta t^{2}-\left(\Delta x^{2}+\Delta y^{2}+\Delta z^{2}\right)
\end{align*}
Notice the composition of the above formula. $\left(\Delta x^{2}+\Delta y^{2}+\Delta z^{2}\right)$ is simply the distance between the 2 events squared, and $\Delta t^{2}$ is the time interval squared. $\Delta S^{2}$ is actually \textbf{not a squared value} as the result of a square operation must be positive and the right hand side of the equation may result in a negative number. The space-time interval is a relativistic invariant, meaning all observers will agree on its value. If another observer S' observes the same 2 events, observes the same difference in space/time, S' will compute $\left(\Delta S'\right)^{2}$ defined in the same way as S from above, but $\left(\Delta S'\right)^{2}=\left(\Delta S\right)^{2}$.
}
\par{As all $\Delta s^{2}$ is invariant, we can classify all pairs of events as any of the three below:
	\begin{itemize}
		\item{If $\Delta s^{2}<0$, the distance between the 2 events is bigger than the time taken, and the events are \textbf{space-like}.}
		\item{If $\Delta s^{2}=0$, the events are \textbf{light-like}, or \textbf{null}.}
		\item{If $\Delta s^{2}>0$, the events are \textbf{time-like}.}
	\end{itemize}
If 2 events are not causally related (observers may argue on which event happened first), then the 2 events must be \textbf{space-like}. Causally related events must be \textbf{time-like} or \textbf{light-like}.

The above formula can be rearranged for the speed of the signal required for the signal to be present at both events described:
\[
c\ \{<, >, =\}\ \frac{\sqrt{\Delta x^2 + \Delta y^{2}+\Delta z^{2}}}{\|\Delta t\|}
\] 

}
\subsection{Light-like events}
\par{2 events are said to be \textbf{connected by a light signal}, if a photon could have been present at both events: e.g if an atom emits a photon and another atom receives that photon. All null events are connected by a light-signal. Proof: For any pair of events connected by a light signal, we must have
	\begin{align*}
		c=\frac{\text{distance between events}}{\text{time between events}} \\
		c=\frac{\sqrt{\Delta x^{2}+\Delta y^{2}+\Delta z^{2}}}{\|\Delta t\|} \\
		c^{2}=\frac{\Delta x^{2}+\Delta y^{2}+\Delta z^{2}}{\left(\Delta t\right)^{2}} \\
		c^{2}\left(\Delta t\right)^{2}=\Delta x^{2}+\Delta y^{2}+\Delta z^{2} \\
		c^{2}\left(\Delta t\right)^{2}-\Delta x^{2}+\Delta y^{2}+\Delta z^{2}=0 \\
	\end{align*}
}
\subsection{Time-like events}
\par{Events are time-like if the space-time interval is positive. An example: S sees 2 events which occur at the same point in space. As they are at the same point, their distance is 0, and as the time component of spacetime interval is the result of a square, it must be positive.
\begin{align*}
	c^{2}\Delta t^{2}>\left(\Delta x\right)^{2}+\left(\Delta y\right)^{2}+\left(\Delta z\right)^{2} \\
	c>\frac{\sqrt{\Delta x^{2}+\Delta y^{2}+ \Delta z^{2}}}{\|\Delta t\|}
\end{align*}
This above shows that it is possible for such event as the speed (distance over a time) is less than the speed of light. This implies that it is possible for an observer to be present at 2 events. That implies that timelike events may be causally related. As the observer can be present at both events, it is possible for the observer to have caused both events. Also implies that all observers also agree on the temporal ordering of the events (1 event must occur after the other). }
\subsection{Space-like events}
\par{Events are space-like if the space-time interval is negative.
\[
\Delta s^{2}=c^{2}\Delta t^{2}-\left(\Delta x^2 + \Delta y^{2}+\Delta z^{2}\right)<0
\] 
For example: On a train car, observer S sees a light bulb shoot 2 photons simultaneously, one towards the front end of the train car, the other to the back. As both occur in 1 instant to the observer, $\Delta t=0$, and the distance between 2 events are $\Delta x\ne0$ As a result, 
\begin{align*}
\Delta s^{2}=0-\Delta x^{2} \\
\implies \Delta s^{2}<0 \\
\implies \Delta s^{2}=c^{2}\Delta t^{2}-\left(\Delta x^2 + \Delta y^{2}+\Delta z^{2}\right)<0 \\
\implies c<\frac{\sqrt{\Delta x^2 + \Delta y^{2}+\Delta z^{2}}}{\|\Delta t\|}
\end{align*}
This is completely impossible as the speed of the signal must be greater than the speed of light, and thus, the 2 events must not be causally related, as the signal cannot travel from 1 event to another in time to catch both events. This implies that space-like events are not causally related. Also implies that different observers can disagree on the temporal ordering of the events. }
\section{Velocity transformations}
\par{If two observers travel at different relativistic speeds (significant fraction of speed of light), they cannot simply add/subtract vectors to determine each other's speeds as per Gallilean relativity. This is because observers cannot observe anything to be moving greater than the speed of light, which Gallilean relativity allows. This can be derived from the spacetime-interval, which all observers agree upon.}
\par{Take 2 observers, S and S', with the coordinates $S=\left(t,x,y,z\right), S'=\left(t',x',y',z'\right)$. S is on a planet, observing 2 spaceships on a collision course, whereas S' is on one of the spaceships. S observes the empty spaceship to be moving at a velocity $\vec{u}=\left(u_{x},u_{y},u_{z}\right)$ and S' observes it to be moving at a velocity $\vec{u'}=\left(u'_{x},u'_{y},u'_{z}\right)$. Both observers view the moving object to be of constant velocity. If one observer views an object to be at constant velocity, then all observers must view it as at constant velocity. If we imagine a spacetime diagram, with 2 events on the spaceships' worldline. S will view the distance and time duration between the 2 events to be: $S:\left(\Delta t, \Delta x, \Delta y, \Delta z\right)$ and $S':\left(\Delta t', \Delta x', \Delta y', \Delta z'\right)$. As a result the 2 velocities will simply be: }
\begin{align*}
	S: \vec{u}=\left(\frac{\Delta x}{\Delta t},\frac{\Delta y}{\Delta t},\frac{\Delta z}{\Delta t}\right) \\
	S': \vec{u'}=\left(\frac{\Delta x'}{\Delta t'},\frac{\Delta y'}{\Delta t'},\frac{\Delta z'}{\Delta t'}\right) \\
\end{align*}
\par{As we have the 2 velocities, we can use the relationships for time dilation and length contraction to find one given we have the other.}

TODO: add velocity transformation derivation here.
\section{Relativistic Energy and Momentum}
\par{We again define and observer S: $\left(t,x,y,z\right)$ viewing an object moving at velocity: $\vec{u}$. Let the velocity of a single \textbf{massive} (travelling at less than the speed of light.) particle be observed by S to be moving at $\vec{u}:\left(\frac{dx}{dt},\frac{dy}{dt},\frac{dz}{dt}\right)=\left(u_{x},u_{y},u_{z}\right)$: and the speed given by $u=\sqrt{\vec{u}\cdot\vec{u}}=\sqrt{u_{x}^{2}+u_{x}^{2}+u_{x}^{2}}$, provided that $u<c$. The (relativistic) momentum of this particle as observed by S is now:}
\begin{align*}
	\vec{p}=\left(p_{x},p_{y},p_{z}\right) \\
	=\frac{m\vec{u}}{\sqrt{1-\frac{v^{2}}{c^{2}}}} \\
	=\gamma \left(u\right)m\vec{u} \\
	\text{where } \gamma\left(u\right)=\frac{1}{\sqrt{1-\frac{v^{2}}{c^{2}}}} \\
\end{align*}
$m=$ Mass of the particle, sometimes called the 'rest mass'. It is an invariant and measured in standard S.I units (Kg).
\subsection{Relativistic energy}
\par{The energy of the particle is given as:}
\begin{align*}
	E=\frac{mc^{2}}{\sqrt{1-\frac{u^{2}}{c^{2}}}} \\
	=\gamma\left(u\right)mc^{2}
\end{align*}
If at rest, the energy is called: $E_{0}$ ('E nought' is typically used to denote at rest), Then the kinetic energy will just be:
\begin{align*}
	K=E-E_{0} \\
	=\gamma\left(u\right)mc^{2}-mc^{2} \\
	=\left(\gamma\left(u\right)-1\right)mc^{2}
\end{align*}
At low speeds: 
\begin{align*}
\gamma\left(u\right)=1+\frac{1}{2}\frac{v^{2}}{c^{2}} \\
k=\left(\gamma\left(u\right)-1\right)mc^{2}\\
=\left(\frac{1}{2}\frac{u^{2}}{c^{2}}\right)mc^{2} \\
=\frac{1}{2}mu^{2}
\end{align*}
Which makes sense in Newtonian physics.
\subsection{Momentum Energy Relation}
\par{We can combine both momentum and energy into 1 equation by eliminating $u$}
\begin{equation*}
	\begin{aligned}[c]
		\vec{p}\cdot\vec{p}&=\gamma\left(u\right)^{2}m^{2}\vec{u}\cdot\vec{u} \\
		p^{2}&=\frac{m^{2}u^{2}}{1-\frac{u^{2}}{c^{2}}}
	\end{aligned}
	\quad\leftrightarrow\quad
	\begin{aligned}[c]
		E&=\gamma\left(u\right)mc^{2} \\
		E^{2}&=\gamma\left(u\right)^{2}m^{2}c^{4} \\
		E^{2}&=\frac{m^{2}c^{4}}{1-\frac{u^{2}}{c^{2}}}
	\end{aligned}
\end{equation*}
\begin{align*}
	p^{2}&-\frac{u^{2}p^{2}}{c^{2}}=m^{2}u^{2} \\
	p^{2}&=m^{2}u^{2}+\frac{u^{2}p^{2}}{c^{2}} \\
	p^{2}&=u^{2}\left(m^{2}+\frac{p^{2}}{c^{2}}\right) \\
	u^{2}&=\frac{p^{2}}{m^{2}+\frac{p^{2}}{c^{2}}} \\ 
	\frac{u^{2}}{c^{2}}&=\frac{p^{2}}{c^{2}m^{2}+p^{2}} \\
	1-\frac{u^{2}}{c^{2}}&=1-\frac{p^{2}}{c^{2}m^{2}+p^{2}} \\
	&=\frac{c^{2}m^{2}+p^{2}-p^{2}}{c^{2}m^{2}p^{2}} \\
	1-\frac{u^{2}}{c^{2}}&=\frac{c^{2}m^{2}}{c^{2}m^{2}+p^{2}} \\
\end{align*}
After substitution and cancelling we get:
\[
E^{2}=p^{2}c^{2}+c^{4}m^{2}
\] 
For a highly relativistic particle:
\begin{align*}
	&u\to c \\
	&\gamma\left(u\right)\to\infty \\
	&E^{2}\to\infty \\
	&p^{2}\to\infty
\end{align*}
Then the constant at the back $m^{2}c^{4}$ is ignorable (implying that its mass is ignorable) and that \[
E=pc
\]  for particles at light speed. As a result, we consider light-speed particles to be mass-less. 
\subsubsection{The Momentum of a Mass-less Particle}
\par{As it is mass-less, it is travelling at the speed of light, $u=c$. We introduce a unit vector. The momentum of a mass-less/light-speed particle is defined as:}
\[
	\vec{p}=\frac{E}{c}\hat{n}
\] 
Note: massive particles can have $\vec{p}=0$ by being at rest, but all particles, whether at rest or otherwise, massive or mass-less, all particles \textbf{MUST} carry non-zero energy. Therefore: mass-less particles cannot have $\vec{p}=0$ as they are moving.

\subsubsection{Conservation Laws:}
\par{For an isolated dynamical system in mechanics where particles interact only by contact (collisions, extremely short-ranged forces), the total relativistic momentum and energy of the system is then conserved (Stays the same as time passes). Total momentum is the sum of all the momenta of particles in the system (sum). The same is true for total energy}
\begin{align*}
	\vec{P}_{T}=P_{1}+P_{2}+P_{3}...P_{n} \\
	\vec{E}_{T}=E_{1}+E_{2}+E_{3}...E_{n} \\
\end{align*}

\chapter{Resonance}
\section{Harmonic Oscillator}
\par{Consider some object dangling from some ceiling from a spring. First, consider the spring to not be in motion, in its equilibrium position, where the sum of the forces it 0, the tension is balanced by the downwards force of gravity. Lets then suppose our coordinate system has the positive x direction going straight up, and the negative x direction going straight down. If we pull the spring back, it will experience a Restoring force, here denoted as: \[
F_{restoring}=-kx
\]
Where the negative sign reminds us that this force acts in the opposite direction of the original force. If we then replace $F$ with its definition of mass times acceleration, we get:  \[
m\frac{d^{2}x}{dt^{2}}=-kx
\] If we ignore the constants for a second, we get the following differential equation: \[
\frac{d^{2}x}{dt^{2}}=-x
\], Which we can guess the solution to be $x=\cos\left(t\right)$. However, this does not include the coefficients $k$ and $m$, which are necessary, instead, we consider a horizontal dilation in the cosine graph.
\begin{align*}
	& m\frac{d^{2}x}{dt^{2}}=-kx \\
	& \frac{d^{2}x}{dt^{2}}=-\frac{k}{m}x \\
	& \text{consider $\omega_{0}$ to be the angular frequency, then:} \\
	& \text{let } x=A\cos\left(\omega_{0}t\right) \\
	& \frac{dx}{dt}=-A\omega_{0}\sin\left(\omega_{0}t\right) \\
	& \frac{d^{2}x}{dt^{2}}=-A\omega_{0}^{2}\cos\left(\omega_{0}t\right) \\
	& \frac{k}{m}=\omega_{0}^{2}
\end{align*}

Here, the omega dilates the time (in the horizontal direction of the graph). The period of a usual cosine function is $2\pi$, and as it is being dilated by a factor of $\omega_{0}, \left(\frac{1}{\omega_{0}}\right)$ then: \[
T=2\pi\sqrt{\frac{m}{k}}
\] What this means physically, is that due to a greater mass, the inertia will be greater, leading to an even greater period. If the spring was stronger, $k$ would be larger, and thus the period shorter, and the time taken, shorter.

The starting position of the spring is dependent on when we first observe the spring. This can mean that there is a horizontal translation, as we may view the spring being launched up, or coming down. \[
x=A\cos\left(\omega_{0}t+\Delta\right)
\] We can rearrange this to be a sum of 2 trigonometric functions: \[
x=A\left(\cos\left(\omega_{0}t\right)\cos\left(\Delta\right)-\sin\left(\omega_{0}t\right)\sin\left(\Delta\right)\right)
\] Where:
\begin{itemize}
	\item{$A$: is the amplitude of the function, or the \textbf{vertical dilation}}
	\item{$\Delta$: is the phase shift, or the \textbf{horizontal translation}}
	\item{$\omega$: is the \textbf{angular frequency}}
\end{itemize}
We can then collapse this by replacing the constants $\cos\left(\Delta\right)$ and $\sin\left(\Delta\right)$ 
\begin{align*}
	& \text{let } B=A\cos\left(\Delta\right) \\
	& \text{let } B=-A\sin\left(\Delta\right) \\
	& x=A\left(\cos\left(\omega_{0}t\right)\cos\left(\Delta\right)-\sin\left(\omega_{0}t\right)\sin\left(\Delta\right)\right) \\
	& x=B\cos\left(\omega_{0}t\right)+C\sin\left(\omega_{0}t\right)
\end{align*}}
\subsection{Initial Conditions}
\par{The values B and C from the previous equation as well as A are whats known as the initial conditions, values that are determined at the first instance of observation.Imagine an initial time, position and velocity. $t=0, x=x_{0}, v=v_{0}$. If we again consider x and v as its derivative, we get: }
\begin{align*}
	x&=B\cos\left(\omega_{0}t\right)+C\sin\left(\omega_{0}t\right) \\
	v=\frac{dx}{dt}&=-\omega_{0}B\sin\left(\omega_{0}t\right)+\omega_{0}C\cos\left(\omega_{0}t\right) \\
\end{align*}
\par{When we plug in our intial conditions of $t=0, x=x_{0}, v=v_{0}$, we reduce the sine and cosine down to 1 and zero: }
\begin{align*}
	& x=B\cos\left(\omega_{0}\left(0\right)\right)+C\sin\left(\omega_{0}\left(0\right)\right) \\
	& v=\frac{dx}{dt}=-\omega_{0}B\sin\left(\omega_{0}\left(0\right)\right)+\omega_{0}C\cos\left(\omega_{0}\left(0\right)\right) \\
	& \implies x=B\left(1\right)+C\left(0\right) \\
    & \implies v=-\omega_{0}B\left(0\right)+\omega_{0}C\left(1\right) \\
	& x=B \\
	& v=\omega_{0}C \\
\end{align*}
\par{As the above are only true during the initial conditions, x and v should be replaced too: }
\begin{align*}
	& x_{0}=B \\
	& v_{0}=\omega_{0}C \\
	& t=0
\end{align*}
\subsection{Conservation of energy}
\par{As we are only considering a frictionless system, energy should be conserved. When given the kinetic and potential energy formulae below:}
\begin{align*}
	& U=\frac{1}{2}kx^{2} \\
	& T=\frac{1}{2}mv^{2}
\end{align*}
\par{We can plug in our subsitute for x and v from above into those two generic formulae:}
\begin{align*}
	& U=\frac{1}{2}kA^{2}\cos^{2}\left(\omega_{0}t+\Delta\right) \\
	& T=\frac{1}{2}m\omega_{0}^{2}A^{2}\sin^{2}\left(\omega_{0}t+\Delta\right)
\end{align*}
\par{Both of these energies are non-constant, but added together, they form a constant total energy.\[
T+U=\frac{1}{2}kA^{2}\cos^{2}\left(\omega_{0}t+\Delta\right) +\frac{1}{2}m\omega_{0}^{2}A^{2}\sin^{2}\left(\omega_{0}t+\Delta\right)
\] And given the earlier established $k=m\omega_{0}^{2}$ \[
T+U=\frac{1}{2}m\omega_{0}^{2}A^{2}\cos^{2}\left(\omega_{0}t+\Delta\right) +\frac{1}{2}m\omega_{0}^{2}A^{2}\sin^{2}\left(\omega_{0}t+\Delta\right)
\] Which can then be factorised: \[
T+U=\frac{1}{2}m\omega_{0}^{2}A^{2}\left(\cos^{2}\left(\omega_{0}t+\Delta\right)+\sin^{2}\left(\omega_{0}t+\Delta\right)\right)
\] and simplified using the trigonometric identity $\cos^{2}\left(x\right)+\sin^{2}\left(x\right)=1$ \[
T+U=\frac{1}{2}m\omega_{0}^{2}A^{2}\left(1\right)
\] And now we can see that the total energy is constant (Conserved) and independent of time. }
\subsection{Forced Harmonic Oscillator}
\par{Forced Harmonic Oscillator refers to a harmonic oscillator that is being affected by an external force. The general differential equation therefore has an extra force term added to it: \[
m\frac{d^{2}x}{dt^{2}}=-kx+F\left(t\right)
\]}
\subsection{Complex Numbers and Harmonic Motion}
\subsubsection{Complex Numbers}
\par{Complex numbers come in the form $z=x+iy, i^{2}=-1$. 'x' is known as the real part, sometimes denoted by $\mathrm{Re}\left[z\right]=x$ and 'iy' is known as the complex component, denoted by $\mathrm{Im}\left[z\right]=iy$. 'z' can also be written in the modulus/argument form, $z=re^{i\theta}$, where 'r' is known as the modulus, $|z|=r$ and  $\theta$ is the argument, denoted $\mathrm{Arg}\left[z\right]=\theta$. 'r' and '$\theta$' can be related through $r^{2}=x^{2}+y^{2}$, which can also be written $r^{2}=\left(x+iy\right)\left(x-iy\right)=zz*$, with the 'z*' being known as the complex conjugate ($\left(x-iy\right)$). It is also good to know that $\tan\left(\theta\right)=\frac{y}{x}$. If we again subsitute x as $\cos\left(\theta\right)$ and y as $\sin\left(\theta\right)$, we get $r\cos\left(\theta\right)+i\sin\left(\theta\right)$. This can give us Euler's formula:}
\begin{align*}
	& z=x+iy \\
	& \implies r\left(\cos\left(\theta\right)+i\sin\left(\theta\right)\right)= re^{i\theta} \\
	& \implies \left(\cos\left(\theta\right)+i\sin\left(\theta\right)\right)= e^{i\theta} 
\end{align*}
\par{Euler's formula allows us to connect exponentials to trigonometric functions, which is extremely useful for solving/simplifying the solving process of differential equations regarding trig functions. }
\subsection{Frictional Terms}
\par{Apart from the force, there is also friction which is affecting the term. Imagine a mass moving horizontally over a puddle of oil. The faster the mass moves, the greater the oil resists, and we say that Friction is \textbf{proportional to the speed}.}
\begin{align*}
 	& m\frac{d^{2}x}{dt^{2}}+kx=F_{applied}+F_{friction} \\
	& F_{friction}=-bv \\
	& F_{friction}=-b\frac{dx}{dt} \\
	& \implies m\frac{d^{2}x}{dt^{2}}+kx=F_{applied}-b\frac{dx}{dt} \\
	& \implies m\frac{d^{2}x}{dt^{2}}+b\frac{dx}{dt}+kx=F_{applied} \\
\end{align*}
\par{Where: 
\begin{itemize}
	\item{k is the spring constant}
	\item{b is the damping constant}
	\item{m is the mass}
	\item{x is displacement and F is force}
\end{itemize}}
\subsubsection{Solving}
\par{To solve this DE, we can consider the trick regarding complex numbers, in order to turn differentiation of trig functions to differentiation of exponential functions. 
\begin{itemize}
	\item{First we simplify the above by dividing through the function with mass: 
			\[
				m\frac{d^{2}x}{dt^{2}}+b\frac{dx}{dt}+kx=F
			\] \[
				\frac{d^{2}x}{dt^{2}}+\frac{b}{m}\frac{dx}{dt}+\frac{k}{m}x=\frac{F}{m}
			\] We let 2 new constants take the place of the ratio between the damping constant and mass, and the ratio between the spring constant and mass $\gamma=\frac{b}{m}$ and as defined earlier $\omega_{0}^{2}=\frac{k}{m}$. \[
				\frac{d^{2}x}{dt^{2}}+\gamma\frac{dx}{dt}+\omega_{0}^{2}x=\frac{F}{m}
			\] 
	 }
	\item{First we consider the displacement and force as complex numbers. This is only a trick, there is no real world meaning of complex numbers.\[
				\frac{d^{2}\hat{x}}{dt^{2}}+\gamma\frac{d\hat{x}}{dt}+\omega_{0}^{2}\hat{x}=\frac{\hat{F}}{m}
		\] We can then consider $\hat{x}$ in the exponential form:  \[
		\hat{x}=\hat{x}_{0}e^{i\omega t}
		\] Then the derivative and 2nd derivative shall be: \[
		\frac{d}{dx}\hat{x}=i\omega\hat{x} 
		\] \[
		\frac{d^{2}}{dt^{2}}\hat{x}=\left(i\omega\right)^{2}\hat{x}	
		\] Finally, we can plug those into our DE and get: \[
		\left(i\omega\right)^{2}\hat{x}+\gamma\left(i\omega\right)\hat{x}+\omega_{0}^{2}\hat{x}=\frac{\hat{F}}{m}
		\]}
	\item{We then rearrange for $\hat{x}$:  \[
				\hat{x}=\frac{1}{m\left(\omega_{0}^{2}-\omega^{2}+i\gamma\omega\right)}\hat{F}
	\] If we consider the coefficient of F hat as some constant $R$, we can calculate the modulus and phase angle of r as $\rho$ and $\theta$\[
	R=\frac{1}{m\left(\omega_{0}^{2}-\omega^{2}+i\gamma\omega\right)}=\rho e^{i\theta}
\] If we put this into our complex equation and replace in what $\hat{F}$ is, \[
\hat{x}=\rho e^{i\omega}\hat{F}=
\hat{x}=\rho e^{i\omega}F_{0}e^{i\omega t}
\] Where $F_{0}$ is the modulus of F and $\omega t$ is the argument. We can then collect the exponentials together: \[
\hat{x}=\rho F_{0}e^{i\left(\theta+\omega t\right)}
\]} 
	\item{When we consider the real component: \[
			\hat{x}=\rho F_{0}e^{i\left(\omega t+\theta\right)}
\] \[
			\hat{x}=\rho F_{0}\cos\left(\omega t+\theta\right)+i\sin\left(\omega t+\theta\right)
\] \[
			\mathrm{Re}\left[\hat{x}\right]=\rho F_{0}\cos\left(\omega t+\theta\right)
	\] From this, we can see that $\rho$ gives us the amplitude of the responding oscillation while $\theta$ gives us the phase shift of the response.}
\end{itemize}}
\subsection{Electrical Resonance}
\par{The differential equation in electrical resonance is: $L\frac{d^{2}q}{dt}+R\frac{dq}{dt}+\frac{1}{C}q=V\left(t\right)$ Consisting of the following: 
\begin{itemize}
	\item{q is charge, analogous to displacement $\left(x\right)$}
	\item{$V\left(t\right)$ is voltage, analogous to force $\left(F\left(t\right)\right)$}
	\item{C is capacitance, analogous to the inverse of the spring stiffness/constant $\frac{1}{k}$}
	\item{R is resistance, analogous to friction, $\left(b\right)$}
	\item{L is inductance, analogous to the mass $\left(m\right)$}
\end{itemize}
It is necessary to note Ohm's law, given normally by: \[
V=IR
\] however, we must formally say that current $I$ is the first order derivative of charge: \[
I=\frac{dq}{dt}
\] therefore, Ohm's law can be written: \[
V=R\frac{dq}{dt}
\]analogous to the previous: \[
F_{friction}=b\frac{dx}{dt}
\] 
}
\subsubsection{Capacitor}
\par{A capacitor is an object with 2 parallel metal plates and when they are connected to a circuit with a current flowing, there will be a charge build up on one plate, with an opposite charge build up on the other plate. The difference in charge causes a potential difference (difference between charge) and an electric field. The potential difference between the plates are given by: \[
V=Ed
\] Where $V$ represents the voltage, E is the electric field between, and d is the distance between the plates. The electric field is simply the charge density divided by the constant $\epsilon_{0}$ \[
V=\frac{\sigma}{\epsilon_{0}}
\] And the charge density is just \[
\sigma=\frac{q}{A}
\] Where A is the area of \textbf{of each plate}. Putting everything together, we can see that the Voltage is determined by the geometry of the plate: \[
V=\frac{qd}{A\epsilon_{0}}
\] However, if we separate q, the coefficient of q is defined as $\frac{1}{C}$  \[
V=\frac{1}{C}q
\]  Where c is the capacitance of the circuit element, rearranged to be: \[
C=\frac{q}{V}
\] and measured in Farads (F).}
\subsubsection{Inductor}
\par{An inductor is a coil, of which is not insulated against itself, so there forms a magnetic field which interacts with the current in the other coils. In alternating current (AC), this induces a back emf, working against the current. The voltage of back emf is \textbf{proportional to the rate of change of the current flowing through it} \[
V=L\frac{dI}{dt}
\] The proportionality constant $L$ is the inductance, in Henrys (H). Inductance refers to the ability of an element to store energy as a magnetic field when a current is passing through it. If we realise current is the derivative of charge, the derivative of current is the 2nd order derivative of charge: \[
V=L\frac{d^{2}q}{dt^{2}}
\] }
\subsection{Kirchoff's Laws}
\subsubsection{Kirchoff's Junction Law}
\par{Kirchoff's Junction law states that: The sum of the currents entering any junction (i.e a point on the circuit where the current can split) in a circuit must equal the sum of currents leaving that junction. \[
\sum_{in}^{}{I}=\sum_{out}^{}{I}
\] This is essentially conservation of charge.}
\subsubsection{Kirchoff's Loop Law}
\par{Kirchoff's Loop law states that: The sum of the charges in electric potential across all elements in a complete circuit loop must be zero. \[
\sum_{loop}^{}{\Delta V}=0
\] This can be written as: \[
V_{total}-V_{1}-V_{2}-V_{3}...-V_{n}=0
\] This is essentially conservation of energy.}
\section{LC Circuit}
\par{Consists of a capacitor and a inductor, and initially, the capacitor is charged. We consider no loss in the circuit (ideal). The LC circuit produces an oscillation of charge, bouncing energy being stored in as potential difference (electrical energy) in the capacitor vs as magnetic field (magnetic energy) in the inductor. This is similar to how in the spring, the energy is being bounced between springs' potential energy and the mass' kinetic energy. 
\begin{itemize}
	\item{Initially, There is a build up of charge on one of the plates.}
	\item{As soon as the inductor is connected, there will be a current flowing through the circuit.}
	\item{The inductor resists change in current, current cannot instantaneously change with an inductor sitting there, so the current builds up}
	\item{As the current builds, the magnetic field induced by the inductor builds, all the way until the capacitor is drained, and all of the capacitor's stored energy is now stored as magnetic energy in the inductor.}
	\item{As the inductor is building, however, there is now a gradual build up of electrons flowing towards the other end of the capacitor. This means as soon as the current starts to slow, the magnetic fields around the inductor weaken. }
	\item{This continues until the inductor no longer has stored energy, and the capacitor is now charged in the opposite way round to the initial state, where the positive side is the opposite of its initial. All the energy is now stored as electrical energy. }
	\item{The circuit will now go the opposite direction, discharging from the capacitor to the inductor in the opposite direction, having the same effect it had the first time around. }
\end{itemize}
From this, we obtain the DE via Kirchoff's loop law: \[
V_{Inductor}+V_{Capacitor}=0
\]. Replacing $V_{Inductor}$ by $\frac{d^{2}q}{dt^{2}}L$ and $V_{Capacitor}$ by $\frac{q}{C}$, we get: \[
\frac{d^{2}q}{dt^{2}}L+\frac{q}{C}=0
\], very similar to the spring version without any friction/damping.From earlier, we know this has the general solution of \[
x=A\cos\left(\omega_{0}t+\Delta\right)
\] replacing displacement with charge, we get: \[
q\left(t\right)=A\cos\left(\omega_{0}t+\Delta\right)
\] Where the natural frequency is given by: $\omega_{0}^{2}\frac{1}{LC}$, or \[
\omega_{0}=\frac{1}{\sqrt{LC}}
\] As we are measuring $q$, the charge of the capacitor, the equation for Voltage follows:  \[
V=\frac{q}{C}
\] and hence the voltage would be: \[
V\left(t\right)=\frac{A}{C}\cos\left(\omega_{0}t+\Delta\right)
\] As current is the first derivative, the formula is: \[
I\left(t\right)=-\omega_{0}A\sin\left(\omega_{0}t+\Delta\right)
\] }
\subsection{Energy of an LC Circuit}
\par{The energy of an LC circuit should stay constant. Remember that energy stored in a capacitor is given: \[
U=\frac{1}{2}CV^{2}
\] $U$ is the energy, $C$ is capacitance and $V$ is voltage. To get the energy stored in an inductor, we first start with the power: \[
P=VI
\] Power is equal to voltage times current. Considering power as the rate of change of energy and Voltage of an inductor as shown earlier to be: $L\frac{dI}{dt}$, we rearrange power to be: \[
P=\left(L\frac{dI}{dt}I\right)
\] \[
\frac{dU}{dt}=\left(L\frac{dI}{dt}\right)I
\] Then we consider the total energy to be: \[
U=\int_{}^{}{\left(L\frac{dI}{dt}\right)I}\mathrm{dt}
\] When the $dt$s cancel and we move the constant L out, we get  \[
U=L\int_{}^{}{I}\mathrm{dI}
\] \[
U=L\frac{1}{2}I^{2}
\] This is analogous to the kinetic energy of the mass. }
\subsection{LCR Circuit}
\par{The standard LC circuit (lacks a motor) follows simple harmonic motion. When we hook it up to an AC voltage source (DC cannot work as there won't be an oscillating 'driving force'), then resonance can occur. In any real world circuit, there will be an internal resistance (Hence the LCR) which dissipates and prevents the amplitude from going to infinity. Given Kirchoff's Loop Law, We know that the Total Voltage should give 0. That means: \[
V\left(t\right)-IR-L\frac{dI}{dt}-\frac{q}{C}=0
\] We represent current as the rate of change of charge: \[
L\frac{d}{dt}\left(\frac{dq}{dt}\right)+R\frac{dq}{dt}+\frac{q}{C}=V\left(t\right)
\] \[
L\frac{d^{2}q}{dt^{2}}+R\frac{dq}{dt}+\frac{q}{C}=V\left(t\right)
\] Remember that the 2nd derivative should not have a coefficient: \[
\frac{d^{2}q}{dt^{2}}+\frac{R}{L}\frac{dq}{dt}+\frac{q}{LC}=\frac{V\left(t\right)}{L}
\] This means that the natural angular frequency is: \[
\omega_{0}=\frac{1}{\sqrt{LC}}
\] When the Voltage function is given (as a sinusoidal function): \[
V\left(t\right)=V_{p}\cos\left(\omega t\right)
\] This is the real component of the complex number \[
V_{p}e^{i\omega t}
\] When we rewrite the equation, we take q as complex as well: \[
\frac{d^{2}\hat{q}}{dt^{2}}+\frac{R}{L}\frac{d\hat{q}}{dt}+\omega_{0}^{2}\hat{q}=\frac{\hat{V}}{L}
\] Due to $\hat{q}=q_{p}e^{i\omega t}$, its derivative is simply: \[
\frac{d\hat{q}}{dt}=q_{p}e^{i\omega t}\times\frac{d}{dt}\left(i\omega t\right)
\] Which simplifies into: \[
\frac{d\hat{q}}{dt}=i\omega q_{p}e^{i\omega t}=i\omega\hat{q}
\] Plugging this into our DE, we get: \[
\left(i\omega \right)^{2}\hat{q}+\frac{R}{L}i\omega\hat{q}+\omega_{0}^{2}\hat{q}=\frac{\hat{V}}{L}
\] Which can be simplified into:} 
\begin{align*}
-\omega^{2}\hat{q}+\frac{R}{L}i\omega\hat{q}+\omega_{0}^{2}\hat{q}=\frac{\hat{V}}{L} \\
\frac{R}{L}i\omega\hat{q}+\omega_{0}^{2}\hat{q}-\omega^{2}\hat{q}=\frac{\hat{V}}{L} \\
\hat{q}\left(\frac{R}{L}i\omega+\omega_{0}^{2}-\omega^{2}\right)=\frac{\hat{V}}{L} \\
\implies\hat{q}=\frac{\hat{V}}{L\left(\omega_{0}^{2}-\omega^{2}+i\frac{R}{L}\omega\right)}
\end{align*}
\par{This is practically the same as the equation obtained from the mechanical model, just with all the electronic analogues. If we re-plug in the definition of the natural frequency:}
\begin{align*}
& \hat{q}=\frac{\hat{V}}{L\left(\omega_{0}^{2}-\omega^{2}+i\frac{R}{L}\omega\right)} \\
& \hat{q}=\frac{\hat{V}}{L\left(\frac{1}{LC}-\omega^{2}+i\frac{R}{L}\omega\right)} \\
& \hat{q}=\frac{\hat{V}}{\frac{L}{LC}-L\omega^{2}+i\frac{LR}{L}\omega} \\
& \hat{q}=\frac{\hat{V}}{\frac{1}{C}-L\omega^{2}+i R\omega} \\
\end{align*}
When We consider the current in this complex domain, it is earlier defined as $i\omega$, hence:
\begin{align*}
& \hat{I}=\frac{i\omega\hat{V}}{\frac{1}{C}-L\omega^{2}+i R\omega} \\
& i\omega\hat{V}=\hat{I}\left(\frac{1}{C}-L\omega^{2}+i R\omega\right) \\
& \hat{V}=\hat{I}\left(\frac{1}{i\omega C}-\frac{L\omega^{2}}{i\omega}+R\right) \\
& \hat{V}=\hat{I}\left(R+i\left(\omega L-\frac{1}{\omega C}\right)\right) \\
& \text{Let }\hat{Z}=\left(R+i\left(\omega L-\frac{1}{\omega C}\right)\right)
\end{align*}
\subsubsection{Complex Impedence}
\par{From the above equation, we can simplify further, down into: \[
		\hat{Z}=R+i\left(\omega L-\frac{1}{\omega C}\right)
\] This is known as complex Impedence and is often used because \[
\hat{V}=\hat{I}\hat{Z}
\] Is very close to $V=IR$, Ohm's law, as well as linking together the capacitance and inductance.}
\subsubsection{Complex Reactance $\left(X\right)$}
\par{Complex impedance has 2 parts, the real part is the resistance, but the complex part is what is known as \textbf{Reactance}. Reactance itself is given by the reactance of the inductor and the reactance of the capacitor. $\omega L$ is the reactance of the inductor, denoted $X_{L}$and $\frac{1}{\omega C}$ is the reactance of the capacitor, denoted $X_{C}$. Therefore impedance gives the generalisation of resistance for capacitors and inductors.\[
		\hat{Z}=R+i\left(X_{L}-X_{C}\right)
\] }
\par{Complex impedance is treated as if it were resistance, but now it applies to inductors and capacitors. We can find the total impedance in both series and parallel circuits, as if they were resistance: i.e: \[
		\hat{Z}_{total_{Series}}=\hat{Z_{1}}+\hat{Z_{2}}+\hat{Z_{3}}+...
\] and \[
\frac{1}{\hat{Z}_{total_{Parallel}}}=\frac{1}{\hat{Z_{1}}}+\frac{1}{\hat{Z_{2}}}+\frac{1}{\hat{Z_{3}}}+...
\]  
\section{Phase Shift and Phasors of Resonance}
\par{Imagine an AC circuit hooked up to a resistor, with voltage $V_{p}\cos\left(\omega t\right)$ and resistance $R$, then current is simply:  \[
I=\frac{V_{p}}{R}\cos\left(\omega t\right)
\] as per Ohm's Law (Impedence for resistors only). Both sinusoidal have the same phase. There is no phase shift between. Neither are leading nor lagging behind the other. 

Now Imagine an AC circuit hooked up to an inductor, with the same input Voltage, and with inductance L. We can then obtain the current (the complex current) with the impedance. Doing so, we get: \[
	\hat{I}=\frac{\hat{V}}{i\omega L}
\] and in polar form: \[
	\hat{I}=-i\frac{V_{p}}{\omega L}e^{i\omega t}
\] When considering the -i, we realised that \[
-i=e^{-i\frac{\pi}{2}}
\] and plugging this in, we get: \[
\hat{I}=e^{-i\frac{\pi}{2}}\frac{V_{p}}{\omega L}e^{i\omega t}
\] as this is an exponential, we can combine the phase angles to give: \[
\theta=\omega t-\frac{\pi}{2}
\] if we then consider the real component of current, we get: \[
I=\frac{V_{p}}{\omega L}\cos\left(\omega t-\frac{\pi}{2}\right)
\] This time, we can see that the current has a phase shift of $\frac{-\pi}{2}$, meaning that it is translated horizontally in the positive x direction (as indicated by the negative). This means that the current runs ahead of the voltage, and is known as \textbf{Voltage leads current by $\frac{\pi}{2}$}.

Finally, if we consider the capacitor, our impedance would be $Z_{C}=-iX_{C}$ and when we find the complex current we get: \[
	\hat{I}=i\omega CV_{p}e^{i\omega t}
\]  and in polar form: \[
	\hat{I}=e^{i\frac{\pi}{2}}\omega CV_{p}e^{i\omega t}
\] So when we consider the real component, the phase shift becomes: \[
\theta=\omega t+\frac{\pi}{2}
\]  here, current is translated horizontally in the negative horizontal direction as indicated by the positive. This is known as \textbf{Voltage lags current by $\frac{\pi}{2}$}.
}
\subsection{Phasors}
\par{Given 2 complex numbers on the Argand diagram: $r_{1}e^{i\theta_{1}}$ and $r_{2}e^{i\theta_{2}}$. If we multiply them together, we get:}
\begin{align*}
r_{1}e^{i\theta_{1}}\times r_{2}e^{i\theta_{2}} \\
=r_{1}r_{2}e^{i\left(\theta_{1}+\theta_{2}\right)}
\end{align*}
\par{This means that we can represent that operation as the phasor where the phase angle is just the sum of the 2 initial phase angles. This then implies that multiplying a complex number has the geometric meaning of rotating the phasor by the phase angle (after scaling the modulus).

Now consider the impedance of a resistor
}




\end{document}

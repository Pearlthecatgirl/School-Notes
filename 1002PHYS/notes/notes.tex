\documentclass[10pt]{report}

\usepackage{amsmath}
\usepackage[margin=1cm,nohead]{geometry}
\author{Pearl Lin 24465993}
\title{Phys Notes}

\begin{document}
\maketitle
\chapter{1 (Special Relativity.)}
\section{Definitions}
\begin{itemize}
	\item{Observer: An observer is an individual armed with measuring equipment like rulers and timers}
	\item{Frame of reference: An object/body or collection of bodies that is considered to be at rest}
	\item{Coordinate system: a "choice" by an observer on how to uniquely label points in space/time in their Frame of Reference}
	\item{Inertial Frame of reference: a frame of reference in which Newton's 1st law holds (inertial means not accelerating! Non inertial means accelerating)(does not accelerate or rotate)}
	\item{inertial observer: an observer that is within an inertial frame of reference}
	\item{inertial coordinate system: a coordinate system within an inertial frame of reference which uses:}
		\begin{itemize}
			\item{an ideal clock for measuring the time}
			\item{a Cartesian (x,y,z) system for locating points in space where things happen}
		\end{itemize}
	\item{an event: something which happens at a particular point within space/particular point in time}
	\item{Space/time coordinate: An inertial observer using an inertial coordinate system (using (x,y,z) Cartesian system and T for time) will assign to every event a single space time coordinate (t,x,y,z)\\ It is also the collection of all possible events, i.e, every (t, x, y, z) is called space-time (Minnoski space)}
	\item{Temporal separation (time interval) between events 1 and 2 is given by: \[
	\Delta t=t_{2}-t_{1}
	\] 
	}
	\item{Distance between events 1 and 2 is given by:
\begin{align*}
	\sqrt{\left(x_{2}-x_{1}\right)^{2}+\left(y_{2}-y_{1}\right)^{2}+\left(z_{2}-z_{1}\right)^{2}} \\=\sqrt{\Delta x^{2}+\Delta y^{2}+\Delta z^{2}}
\end{align*}
	}
\end{itemize}
\section{The Principle of Relativity}
\begin{itemize}
	\item{The laws of physics are the same in all inertial reference frames}
	\item{The outcome of any experiment is the same when performed with an identical initial conditions relative to any inertial frame of reference}
\end{itemize}
\subsection{relative quantities/invariant quantities}
\par{A relative quantity is a physical quantity which different observers may measure to have different values (e.g I am on a train with a ball moving at constant velocity. I observe that the ball is not moving (v=0) hence its kinetic energy is 0. Another observer outside of the train observes the ball to be moving, hence a non-zero kinetic energy.) In contrast, invariant quantities have the same value for all observers (e.g, mass)}
\subsection{Postulates of \textbf{Special Relativity}}
\par{The 2 postulates of \textbf{Special Relativity} are as follows:
\begin{itemize}
	\item{Postulate 1: The principle of relativity: The laws of physics are the same in all \textbf{non inertial frames of reference}}
	\item{Postulate 2: The speed of light: In a vacuum, in all inertial frames, light propagates in a straight line at the same speed, denoted \textit{c} at all times and in all directions. The speed of light is invariant}
\end{itemize}

}

\end{document}

